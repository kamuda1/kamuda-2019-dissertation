\chapter{Uranium Enrichment Regression Results and Discussion}

% https://www.ncbi.nlm.nih.gov/pmc/articles/PMC5554294/

% great examples of various enriched U with a large range of detector resolutions https://www.lanl.gov/orgs/n/n1/docs/la_14206.pdf 

% Although mass spectrometry leads to very accurate analytical results, it has the drawbacks of high costs of the device and its operation involving a long process and destructive sample analysis [1], [2]. https://www.sciencedirect.com/science/article/pii/S1738573317308057#bib1

Verifying the enrichment of HEU through passive nondestructive analysis is important for safeguards applications and homeland security tasks. Nondestructive analysis is preferred to preserve forensics evidence and to allow for remote measurements. 

Measuring the enrichment of uranium is made difficult by the fact that it's characteristic gamma-rays are easily shielded. This task is made more difficult for treaty verification by the restriction placed on inspectors. IAEA inspectors face two major restrictions: a limited amount of time to measure data from declared items and the requirement for an information barrier. This means that direct viewing and analysis of the gamma-ray spectrum - necessary for the enrichment meter method - is impossible. 

The traditional method for using NaI detectors to measure uranium enrichment is the enrichment meter method \cite{Reilly1970}. This method exploits the proportionality between the activity of the 186 keV photon and the enrichment of $^{235}$U. This method requires a 

"Generally, low resolution measurements of 'clean' uranium can be made to 1\% precision over nearly the entire range of uranium enrichment. Measurements in both Europe and the Former Soviet Union have observed bias effects of 5-10\% in low-resolution measurements caused by minor isotopes of uranium. This level of bias has been deemed unacceptable, causing some inspectors to resort to more expensive, time-consuming alternatives like mass spectrometry or liquid-nitrogen-cooler high-resolution detectors. As the international safeguards community attempts to inspect more facilities with less resources, these alternatives become highly undesirable. The minor isotopes come from the use of uranium recycled from reactors being used as feed in the enrichment plants. Daughters of $^{232}$U or $^{236}$U include the thorium decay chain which emits a 238.6-keV gamma ray from $^{212}$Pb. This gamma ray falls in the ROI used to estimate the Compton continuum under the 186-keV gamma ray. In addition, the multitude of high-energy gamma rays from the daughters of $^{232}$U change the shape of the Compton continuum that lies under the desired 186-keV peak. Just to make the entire issue more challenging, the level of this interference varies widely among the samples generally offered to the inspector, causing unpredictable, wide variations in the bias effects" \cite{SPRINKLE1997}

 \cite{VESTERLUND2013}.

Accuracies of +/- 10\% are expected for quick checks of high enriched material while accuracies of +/- 1\% are expected to verify mass spectroscopy measurments \cite{Kull1974}.

% Need to address sample wall effect

NaIGEM code to \cite{MORTREAU2004} 


Changes in background in the facility may affect performance. Studies done measuring uranium enrichment in marine environments have shown that background radiation is important and that existing methods  \cite{Hofstetter2008}.


\begin{equation} \label{eq:uenrichment}
softmax(z_j) = \frac{\exp(z_j)} {\sum_{k=1}^{K} \exp(z_k)}.
\end{equation}

\begin{figure}[H]
	\centering
	\includegraphics[width=0.99\linewidth]{images/GADRAS_enrichment_hofstetter}
	\caption{Uranium isotopic enrichments calculated from GADRASw analysis of HPGe and NaI spectra with and without additional uranium X-ray component. Reproduced from \cite{Hofstetter2008}}.
	\label{fig:GADRAS_enrichment_hofstetter}
\end{figure}


\section{Problem Description and Training Dataset Overview}

The gamma-ray signatures from metallic HEU come from the decay of $^{238}$U, $^{235}$U, $^{232}$U, the daughters of these isotopes, and uranium K x-rays. In low-resolution detectors, The primary photopeaks of $^{235}$U signature are at 144 keV, 163 keV, and 186 keV. $^{238}$U daughters produces photopeaks at 92 keV and 1001 keV. The uranium K x-rays at 98 keV are driven by the alpha decay from $^{234}$U \cite{Hofstetter2008}. $^{234}$U naturally occurs in uranium ore and co-enriches with $^{235}$U due to their close atomic mass. Due to a small branching ratio, $^{234}$U has an unresolvable photopeak at 121 keV in HEU spectra measured by NaI(Tl) detectors. Because of this, the gamma-ray spectrum from $^{234}$U decay is not included in the template spectra.

% FRAM application to U and Pu isotopics https://www.lanl.gov/orgs/n/n1/appnotes/LA-14018-M.pdf

\subsection{Training Datasets}



The software package RadSrc was used to generate the specific gamma-ray activities for $^{235}$U, $^{238}$U, and $^{232}$U templates \cite{Hiller2007}. RadSrc, developed at Lawrence Livermore National Laboratory, uses Bateman equations to calculate in-growth of daughters and their respective specific gamma-ray activities. The specific activities for each species are found in Table \ref{table:specific_activities_radsrc}. The specific activity for the x-ray component was fit by hand to a 93\% enriched HEU spectrum. It is assumed that the x-ray component is driven by alpha decay from $^{234}$U and that $^{234}$U co-enriches linearly with $^{235}$U. Because of this, when simulating additional spectra, it was assumed that the x-ray component would scale linearly with enrichment. One study estimated that impurities of $^{232}$U with a mass fraction of 6.0E-9 existed in one sample of HEU created with recycled material  \cite{RawoolSullivan2012}. Based on this, to account for various burn-up conditions in the recycled material, each sample that included $^{232}$U used a mass fraction randomly chosen between 1e-10 and 1e-8. The probability that a spectrum was simulated with $^{232}$U was one half. 


\begin{table}[H]
\centering
\caption{Specific activities for 50 year old uranium species.}
\label{table:specific_activities_radsrc}
\begin{tabular}{cc}
% \cline{2-3}
% Isotope & photos/second/gram \\ \hline
% Isotope & $\frac{photos}{s g}$ \\ \hline
Isotope & Specific Activity  \\ \hline
\multicolumn{1}{c}{$^{232}$U} & 1.20e12 \\ 
\multicolumn{1}{c}{$^{235}$U} & 2.07e5 \\
\multicolumn{1}{c}{$^{238}$U} & 3.80e3 \\ 
\multicolumn{1}{c}{x-ray} & 4.30e4 \\ \hline
\end{tabular}
\end{table}


\begin{table}[H]
\centering
\caption{Range of parameters used for the uranium enrichment dataset.}
\label{table:hyperparameter_dataset_full_parameters_enrichment}
\begin{tabular}{ccc}
% \cline{2-3}
 & Parameter Range & Sampling \\ \hline
\multicolumn{1}{c}{Source-Detector Distance {[}cm{]}} & 50.5, 175.0, 300 & Uniform \\ % \hline
\multicolumn{1}{c}{Source-Detector Height {[}cm{]}} & 50, 100.0, 150 & Uniform \\ % \hline
\multicolumn{1}{c}{FWHM 662 keV {[}s{]}} & 7.0 & N/A \\ % \hline
% \multicolumn{1}{c}{\begin{tabular}[c]{@{}c@{}}Shielding\\ (Percent 200 keV Attenuated)\end{tabular}} & 0\%, 20\%, 40\%, 60\% & Uniform \\ % \hline
\multicolumn{1}{c}{Integration Time {[}s{]}} & 30 - 3600 & Uniform \\ % \hline
\multicolumn{1}{c}{Linear Calibration Offset} & 0 - 10 & Uniform \\ % \hline
\multicolumn{1}{c}{Calibration Gain} & 0.8 - 1.2 & Uniform \\ % \hline
\multicolumn{1}{c}{Background Counts per Second} & 150 - 250 & Uniform \\ % \hline
\multicolumn{1}{c}{Signal to Background Ratio} & 0.1 - 4.0 & Uniform \\ \hline
\end{tabular}
\end{table}


% \subsection{Autoencoder Performance - Fine Tuning networks on Urban Source Search Versus Uranium Enrichment Datasets}

% This section will compare performance differences when fine tuning an autoencoder using features found from the urban source search and the uranium enrichment datasets. The task of isotope identification in the urban source search dataset is different from the task of quantifying uranium enrichment in the uranium enrichment dataset. If the features found from the urban source search are useful as pretraining for uranium enrichment, pretrained networks may be distributed to the community.


% networks and as feature extractors for a DNN. To do this, a trained CAE and DAE will be connected to a dense network. These will be trained to either fixing the autoencoder's weights or by fine-tuning them while the network learns.


\section{Generalization Results - Simulated Data}

These sections describe how each model performed on simulated enriched uranium.


\subsection{Generalization Results on Shielding}

Enriched uranium spectra were simulated with various amounts of shielding.



\begin{figure}[H]
     \centering
     \begin{subfigure}[b]{0.9\textwidth}
         \centering
         \includegraphics[width=\textwidth]{images/results_easy_distance_comparison}
         \caption{Simple Dataset.}
         \label{fig:results_full_background_inject_simple}
     \end{subfigure}

     \begin{subfigure}[b]{0.9\textwidth}
         \centering
         \includegraphics[width=\textwidth]{images/results_easy_distance_comparison}
         \caption{Complete Dataset.}
         \label{fig:results_full_background_inject_full}
     \end{subfigure}
        \caption{Absolute error for all models trained on the complete dataset. Datasets included here used various amounts of shielding.}
        \label{fig:results_full_background_inject}
\end{figure}



\subsection{Generalization Results on Changing Calibration}

Unshielded enriched uranium spectra were simulated with various changes in their gain.  


\begin{figure}[H]
     \centering
     \begin{subfigure}[b]{0.9\textwidth}
         \centering
         \includegraphics[width=\textwidth]{images/results_easy_distance_comparison}
         \caption{Simple Dataset.}
         \label{fig:results_full_background_inject_simple}
     \end{subfigure}

     \begin{subfigure}[b]{0.9\textwidth}
         \centering
         \includegraphics[width=\textwidth]{images/results_easy_distance_comparison}
         \caption{Complete Dataset.}
         \label{fig:results_full_background_inject_full}
     \end{subfigure}
        \caption{Absolute error for all models trained on the complete dataset. Datasets included here used various amounts of shielding.}
        \label{fig:results_full_background_inject}
\end{figure}


\subsection{Generalization Results on Changing Background}

Unshielded enriched uranium spectra were simulated with various background templates.  


\begin{figure}[H]
     \centering
     \begin{subfigure}[b]{0.9\textwidth}
         \centering
         \includegraphics[width=\textwidth]{images/results_easy_distance_comparison}
         \caption{Simple Dataset.}
         \label{fig:results_full_background_inject_simple}
     \end{subfigure}

     \begin{subfigure}[b]{0.9\textwidth}
         \centering
         \includegraphics[width=\textwidth]{images/results_easy_distance_comparison}
         \caption{Complete Dataset.}
         \label{fig:results_full_background_inject_full}
     \end{subfigure}
        \caption{Absolute error for all models trained on the complete dataset. Datasets included here used various background templates.}
        \label{fig:results_full_background_inject}
\end{figure}




\section{Results - Rocky Flats Shells}


% Is it necessary to re-run the hyperparameter search, or are the 'simple' models good enough?

The reported Rocky Flats shells isotopics are shown in Table \ref{table:rockyflats_isotopics}. Recent work has found estimated that impurities of $^{232}$U, mass fraction 6.0E-9, and $^{237}$Np, mass fraction 1.8E-4, \cite{RawoolSullivan2012} existed in the shells in 1971 (the year the shells were fabricated).

\begin{table}[H]
\centering
\caption{Rocky Flats isotopics \cite{Rothe1997}.}
\label{table:rockyflats_isotopics}
\begin{tabular}{|c|c|}
\hline
Uranium Isotope & \begin{tabular}[c]{@{}c@{}}Weight Percentage\\ (1965)\end{tabular} \\ \hline
$^{233}$U & - \\ \hline
$^{234}$U & 1.00 \\ \hline
$^{235}$U & 93.19 \\ \hline
$^{236}$U & 0.4 \\ \hline
$^{238}$U & 5.41 \\ \hline
\end{tabular}
\end{table}

Asymptotic models are 


Again, measure asymptotic MSE convergence.


\section{Discussion and Conclusion}

Other cases need to be added to the training dataset to make this work in practice. Additional examples need to $^{241}$Am, $^{239}$Pu, and $^{237}$Np from recycled uranium. 





