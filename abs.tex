Current radioisotope identification devices struggle to identify and quantify isotopes in low-resolution gamma-ray spectra in a wide range of realistic conditions. Trained gamma-ray spectroscopists typically rely on intuition when identifying isotopes in spectra. A trained gamma-ray spectroscopist can inject their intuition into pattern recognition algorithms by creating training datasets and intelligently choosing a machine learning model for a task. Algorithms based on feature extraction such as peak finding or ROI algorithms work well for well-calibrated high resolution detectors. For low-resolution detectors, it may be more beneficial to use algorithms that incorporate more abstract features of the spectrum. To investigate this, we simulated datasets and used them to train artificial neural networks (ANNs) for identification and quantification tasks using gamma-ray spectra. Because the datasets were simulated, this method can be extended to a variety of gamma-ray spectroscopy tasks. Models we investigated include dense, convolutional, and autoencoder ANNs. In this work we introduce \verb|annsa|, an open source Python package capable of creating gamma-ray spectroscopy training datasets and applying machine learning models to solve spectroscopic tasks. In this work we demonstrate \verb|annsa|'s capabilities on a source interdiction classification and uranium enrichment quantification problem.
